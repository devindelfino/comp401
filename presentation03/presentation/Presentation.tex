\documentclass[13pt]{beamer}
\usepackage{graphicx}
\usepackage[utf8]{inputenc}
\usepackage[skip=2pt,font=scriptsize]{caption}
\usepackage{algorithm}
\usepackage{algorithmic}

% Algorithms
\renewcommand{\algorithmicrequire}{\textbf{Summary:}}
\renewcommand{\algorithmicensure}{\textbf{Output:}}
\algsetup{linenosize=\small}

% Captions
\captionsetup{labelformat=empty,labelsep=none}
\DeclareCaptionFormat{myformat}{#3}
\captionsetup[algorithm]{format=myformat}

% References
\usepackage{url}
\bibliographystyle{acm}

\setbeamertemplate{bibliography item}[triangle]

% Formatting
\usetheme{Singapore}
\usecolortheme{whale}

% Title Page
\title{The Sieve of Eratosthenes using MPI}
\author{Devin Delfino}
\institute{Comp 401: Senior Seminar}
\date{4/16/2015}

% Table of Contents
\setbeamertemplate{section in toc}[sections numbered]
\setbeamercolor{alerted text}{fg=blue}
\AtBeginSection[]
{
  \begin{frame}
    \frametitle{Outline}
    \tableofcontents[currentsection]
  \end{frame}
}

\begin{document}
% TITLE ------------------------------------------------
\frame{\titlepage}


% Table of Contents ------------------------------------------------
\begin{frame}
\frametitle{Outline}
\tableofcontents
\end{frame}

\section{Parallel Programming} % ==========================================================================
% Parallel Programming Introduction ------------------------------------------------
\begin{frame}
\frametitle{Introduction}

\begin{itemize}
  \item \alert{Parallel Computing} is the use of multiple computers or processors to reduce the time needed to solve a single computational problem.
  \item A \alert{task} is a single program including local memory and a collection of input/output ports.
  \item A \alert{channel} is a message queue between two tasks used for communication
\end{itemize}
\end{frame}
% share definitions and talk about analogy
% workers = processors, job = program, tasks = tasks
% sequential -> assign the entire job to one person
% parallel -> divide the job into various tasks, assign the t tasks to p workers (maximize efficiency when t = p)
%          -> communication may be needed between the workers, representing comm. channels

% Foster's Design: Partitioning ------------------------------------------------
\begin{frame}
\frametitle{Ian Foster's Design Methodology}
  \begin{enumerate}
    \item \alert{Partitioning} - the process of dividing the computations and data into pieces.
    \item \alert{Communication} - channels between tasks allow communication between them
          \begin{itemize}
            \item Local - a task's computation requires values from a small number of other tasks
            \item Global - many tasks must contribute values to perform a computation
          \end{itemize}
    \item \alert{Agglomeration} - grouping tasks in order to improve performance and reduce overhead.
    \item \alert{Mapping} - assigning processes or tasks to specific processors or computers
  \end{enumerate}
\end{frame}

\section{Message Passing Interface (MPI)} % ==========================================================================
% MPI Introduction ------------------------------------------------
\begin{frame}
\frametitle{Message Passing Interface (MPI)}

\end{frame}

% Parallel Programming Design ------------------------------------------------
\begin{frame}

\end{frame}

\section{The Sequential Algorithm} % ==========================================================================
% The Sequential Sieve ------------------------------------------------
\begin{frame}
\frametitle{The Sequential Algorithm}
  \begin{algorithm}[H]
        \caption{The Sieve of Eratosthenes}
        \begin{algorithmic}
          \REQUIRE Finds all primes between $2$ and $n$, inclusive
          \STATE Create a list of natural numbers $2$, $3$, ... , $n$, none of which are marked
          \STATE Set $k$ equal to the first prime number, $2$
          \WHILE{$k^2 \leq n$}
            \STATE Mark all multiples of $k$ between $k^2$ and $n$
            \STATE Set $k$ to the smallest unmarked number greater than the current $k$
          \ENDWHILE
        \end{algorithmic}
        \end{algorithm}
\end{frame}
% Introduce steps
% SHOW ANIMATION

\section{The Parallel Algorithm} % ==========================================================================
% Block Allocation ------------------------------------------------
\begin{frame}
\frametitle{The Parallel Algorithm: Block Allocation}
  \begin{itemize}
    \item The main question is how to break up the problem into multiple tasks
  \end{itemize}
\end{frame}

% Translating Sequential Algorithm ------------------------------------------------
\begin{frame}
\frametitle{Developing the Algorithm}
  
\end{frame}

\section{Sequential vs. Parallel Comparison} % ========================================================================
% Comparison: Seq vs. Par ------------------------------------------------
\begin{frame}
\frametitle{Sequential vs. Parallel}
  
\end{frame}

% References ------------------------------------------------
 \begin{frame}
  \frametitle{References}
  \nocite{*} 
  \bibliography{workscited}
\end{frame}

\end{document}