\documentclass[13pt]{beamer}
\usepackage{graphicx}
\usepackage[utf8]{inputenc}
\usepackage[skip=2pt,font=scriptsize]{caption}

% Captions
\captionsetup{labelformat=empty,labelsep=none}

% References
\usepackage{url}
\bibliographystyle{acm}
\setbeamertemplate{bibliography item}[triangle]

% Formatting
\usetheme{Singapore}
\usecolortheme{whale}

% Title Page
\title{Geometric Manifolds and the Shape of Space}
\author{Devin Delfino}
\institute{MATH 331: Geometry}
\date{12/02/2014}

% Table of Contents
\setbeamertemplate{section in toc}[sections numbered]
\setbeamercolor{alerted text}{fg=blue}
\AtBeginSection[]
{
  \begin{frame}
    \frametitle{Outline}
    \tableofcontents[currentsection]
  \end{frame}
}

\begin{document}
% TITLE ------------------------------------------------
\frame{\titlepage}

% Table of Contents ------------------------------------------------
\begin{frame}
\frametitle{Outline}
\tableofcontents
\end{frame}

\section{Geometric 2-Manifolds} % ==========================================================================
% 2-MANIFOLDS ------------------------------------------------
\begin{frame}
\frametitle{Introduction}
	\begin{itemize}
		\item A \alert{Geometric 2-Manifold} is a connected surface that is locally isometric to either the Euclidean plane, hyperbolic plane, or sphere.
		\item Cones (excluding cone point), Cylinders, and Tori are examples of flat 2-manifolds
    % \item The double torus is a 2-manifold that is locally isometric to the hyperbolic plane
	\end{itemize}
	\begin{columns}[r] % the "c" option specifies center vertical alignment
    \column{1\textwidth} % column designated by a command
     \centering
      % \begin{figure}
      %   \includegraphics[height=2cm]{./img/torus}
      %   \caption{http://commons.wikimedia.org/wiki/File:Simple\_Torus.svg}
      % \end{figure}
  \end{columns}
	%torus, hyperbolic pants, double torus
\end{frame}

% CONCLUSION ------------------------------------------------
\begin{frame}
\frametitle{Conclusion}
  \begin{itemize}
    \item The orientation of the temperature patterns of these sets hint at the gluings of the 3-Manifold Universe, and ultimately the geometric structure.
    \item NASA's Wilkinson Microwave Anisotropy Probe (WMAP) mapped out the CMB and its temperature patterns from 2001 to 2010.
    \item Results showed that Universe should hold properties similar to Euclidean Geometry
  \end{itemize}

  \begin{columns}[c] % the "c" option specifies center vertical alignment
      \column{1\textwidth}
       \centering
        % \begin{figure}
        %   \includegraphics[height=3.5cm]{./img/cmbsky} 
        %   \caption{http://map.gsfc.nasa.gov/media/121238/index.html}
        % \end{figure}
    \end{columns}
\end{frame}

% QUESTIONS ------------------------------------------------
\begin{frame}
\frametitle{Questions?}
   \begin{columns}[c] % the "c" option specifies center vertical alignment
    \column{.5\textwidth}
     \centering
      % \begin{figure}
      %   \includegraphics[height=4cm]{./img/3torusEarth} % shape of space cover
      %   \caption{http://web.math.princeton.edu/\\conference/Thurston60th/lectures.html}
      % \end{figure}
    \column{.5\textwidth}
     \centering
      % \begin{figure}
      %   \includegraphics[height=4cm]{./img/poincarespaceEarth} % wraparound universe cover
      %   \caption{http://quibb.blogspot.com/2011/05/\\manifolds-shape-of-universe-iii.html}
      % \end{figure}
  \end{columns}
\end{frame}

% % References ------------------------------------------------
%  \begin{frame}[shrink=30]
%   \frametitle{References}
%   \nocite{*} 
%   \bibliography{workscited}
%   % \bibliography{ShapeOfSpace,WraparoundUniverse,ExperiencingGeo,NASA}
% \end{frame}

\end{document}